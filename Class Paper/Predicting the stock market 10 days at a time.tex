\documentclass[conference]{IEEEtran}

\usepackage{cite}
\usepackage{amsmath,amssymb,amsfonts}
\usepackage{algorithmic}
\usepackage{graphicx}
\usepackage{textcomp}
\usepackage{xcolor}
\def\BibTeX{{\rm B\kern-.05em{\sc i\kern-.025em b}\kern-.08em
    T\kern-.1667em\lower.7ex\hbox{E}\kern-.125emX}}
\begin{document}

\title{Predicting the Stock Market One Quarter at a Time\\
{\footnotesize \textsuperscript{*}Note: Sub-titles are not captured in Xplore and
should not be used}

}

\author{\IEEEauthorblockN{1\textsuperscript{st} Cory Carter}
\IEEEauthorblockA{\textit{Computer Science Master of Science Student} \\
\textit{Oakland University}\\
Auburn Hills, USA \\
email address or ORCID}
\and
\IEEEauthorblockN{2\textsuperscript{nd} Tom Tisdall}
\IEEEauthorblockA{\textit{Software Engineering and IT Master of Science Student} \\
\textit{Oakland University}\\
Auburn Hills, USA \\
ttisdall@oakland.edu}


}

\maketitle

\begin{abstract}
Predicting the stock market is a very challenging task, with hundreds if not thousands of papers addressing this problem.  This paper sets out to explore predictingstock quarterly trends by determining the best quarterly features to use and then leveraging them to determine the direction of the upcoming quarterly report. The thought was to eventually marry this will daily trends to improve accuracy of predicting stock market pricess by factoring in the quarterly direction.  By determining the stocks that will go up or down for a quarter, we will predict whether buying or selling is the best course of action for a stock. Testing for accuracy, we will simuilate buying and selling in the stock market based on our predictions to determine if this project was a succsess or failure.  

**Add more details to this section as we further develop our strategy
\end{abstract}

\begin{IEEEkeywords}
Index Terms - Stock Market, Prediction, Machine Learning
\end{IEEEkeywords}

\section{Introduction}
Determining the direction of a stock from day to day is both a challenging and frustrating problem.  There are so many features to choose from and that factor into a stock changing price from day to day. It can be as simple as positive or negative news, changes in the overall market or industry, technological breakthroughs, production hurdles and many more, which don't even cover the technical or quantative numbers that typically are used to determine a stock rating and predicting its movement from day to day.  This paper is goign to attempt a different tact on predicting stocks, at a more generic and higher level, using the companies quarterly returns.  The thought is this information can be added to other predictors to better refine the accuracy of those models.  In this project we have gathered twenty five Technology, Software and IT Services large and mega cap stocks with earnings history for their entire existence, typically between ten and thirty years.  We started with the features used in the paper Stock Market Trends Prediction after Earning Release  \cite{QuarterlyPrediction}.  We reviewed and refined those quarterly features by using Mutual Info Regression over the datasets that we obtained.

adding second\cite{DailyReturnDirection}
adding third ref\cite {EmpiricalStudy}
\section{Review of existing techniques for predicting the stock market}


\bibliographystyle{IEEEtran}
\bibliography{Bibliography}


\vspace{12pt}
\color{red}
IEEE conference templates contain guidance text for composing and formatting conference papers. Please ensure that all template text is removed from your conference paper prior to submission to the conference. Failure to remove the template text from your paper may result in your paper not being published.

\end{document}
